\documentclass[a4paper, 12pt]{article}

\usepackage[utf8]{inputenc}
\usepackage[T1]{fontenc}
\usepackage[francais]{babel}
\usepackage{verbatim}
\usepackage{moreverb}
\usepackage{algorithm, algpseudocode}
\usepackage{listings}
\usepackage{listingsutf8}

\renewcommand{\lstlistingname}{Interface}

\lstset{
language=Caml,        % choix du langage de programmation 
  frame=single,
  basicstyle=\small\normalfont\sffamily,    % the size of the fonts that are used for the code
  stepnumber=1,                           % the step between two line-numbers. If it is 1 each line will be numbered
 % numbersep=10pt,                         % how far the line-numbers are from the code
  tabsize=1,                              % tab size in blank spaces
  breaklines=true,                        % sets automatic line breaking
  captionpos=t,                           % sets the caption-position to top
  mathescape=true,
  %stringstyle=\color{white}\ttfamily, % Farbe der String
  showspaces=false,           % Leerzeichen anzeigen ?
  showtabs=false,             % Tabs anzeigen ?
  xleftmargin=17pt,
  framexleftmargin=17pt,
  framexrightmargin=17pt,
  framexbottommargin=5pt,
  framextopmargin=5pt,
  showstringspaces=false,
  inputencoding=utf8,
  extendedchars=true,
  literate=%
            {é}{{\'{e}}}1
            {è}{{\`{e}}}1
            {ê}{{\^{e}}}1
            {ë}{{\¨{e}}}1
            {û}{{\^{u}}}1
            {ù}{{\`{u}}}1
            {â}{{\^{a}}}1
            {à}{{\`{a}}}1
            {î}{{\^{i}}}1
            {ô}{{\^{o}}}1
            {ç}{{\c{c}}}1
            {Ç}{{\c{C}}}1
            {É}{{\'{E}}}1
            {Ê}{{\^{E}}}1
            {À}{{\`{A}}}1
            {Â}{{\^{A}}}1
            {Î}{{\^{I}}}1  
 }

\usepackage{graphicx}
\usepackage{appendix}
\usepackage{lmodern}
\usepackage{fancyhdr}
\usepackage[hidelinks]{hyperref} 
\usepackage{url} 
\usepackage[top=2cm, bottom=2cm, left=2cm, right=2cm]{geometry}
\renewcommand{\thesection}{\arabic{section}}
\usepackage{amsmath}
\usepackage{amssymb}
\usepackage{mathrsfs}
\pagestyle{fancy}
\renewcommand{\headrulewidth}{0pt}
\renewcommand{\footrulewidth}{1pt}
\fancyhead[L,R]{}
\fancyfoot[R]{\textbf{page \thepage}} 
\fancyfoot[L]{Projet Web}
\fancyfoot[C]{}  

\begin{document}
	\begin{titlepage}
			\includegraphics[scale=0.25]{Logo_transparent.png} 
			\begin{center}
				\vspace*{6cm}
				{ \huge Rapport \\
				Projet Web ENSIIE 1A 2018 \\
				\vspace*{1cm}
				\textsc{twitIIE}}
				%\vspace*{2cm}\textbf{}}
			\vspace*{3cm}
				\begin{center} \large
					\textsc{Chamayou} \textsc{Julien}\\
					\textsc{Mauillon} \textsc{Edwin}\\
					\textsc{XU} \textsc{Jiahui}\\
					\textsc{XU} \textsc{Kevin}		
				\end{center}
				\begin{minipage}{0.4\textwidth}
				
				\end{minipage}

				{\large 22 mai 2018}
			\end{center}
	\end{titlepage}


	\renewcommand{\contentsname}{Sommaire} 
	{\setlength{\baselineskip}{1.2\baselineskip}
\tableofcontents\par}
	
	\addcontentsline{toc}{part}{Préambule} %Ajouter un lien
	\newpage
	\vspace*{3cm}
	\paragraph{\Huge{Préambule}}

	\paragraph{}
 Après analyse de toutes les opportunités d'application, nous avons choisi de développer une application de type Twitter. En effet, meme si celle-ci n'apparait pas comme étant la plus simple à créer c'est l'application qui nous parle le plus car nous sommes tous utilisateur de ce réseau-social.
	    Le fait d'utiliser régulièrement cette application nous a permis de tout de suite cibler les services à offrir aux utilisateur.
		L'objectif de notre projet est donc de réaliser une application Web qui va tout copier de twitter et qui va leur voler tout le marché. Nous devons alors intégrer dans notre application les fonctionnalités suivantes : 
\begin{itemize}
\item Inscription et Connexion d'utilisateur
\item Publication de tweet 
\item Modification du profil de l'utilisateur
\item Hashtag et noms d'utilisateur cliquables
\item Commentaires de tweet et de commentaires
\item Possibilité de like des tweet
\item Envoi de messages privés
\end{itemize}


%\section{Présentation}	

Le rapport du projet a un but complètement différent de la soutenance. La soutenance nous permet de voir le produit tel que vous avez imaginé son utilisation, le rapport nous permet de comprendre la partie interne du fonctionnement de votre groupe. Il est très important pour nous de bien comprendre quels ont été les enjeux de chaque groupe, quelles ont été les difficultés rencontrées et surtout quels ont été les solutions trouvées pour les contourner.


Le rapport
Faire 10 pages maximum pour refléter le monde de l’entreprise où la concision est une qualité
Expliquer l’approche mise en place, les problématiques rencontrées (techniques comme méthodes) et les solutions apportées
Expliquer la répartition des rôles au sein de l’équipe
La problématique à laquelle il répond.

\newpage

\section{La connexion}	
L'application web TwitIIE est basée sur le fait même d'avoir un compte par personne, éventuellement un compte pour un groupe ou une association. Il est donc nécessaire de créer cette page de connexion, d'une part pour pouvoir accéder aux fonctionnalités de l'application mais également pour pouvoir authentifier les tweets envoyés.
La page de connexion est donc la première page présentée à l'utilisateur. Pour pouvoir accéder à son compte il est nécessaire de remplir un formulaire contenant : 
\begin{description}
\item[Login :] Un login, propre à chacun et unique qui permet l'authentification de chaque personne.
\item[Mot de passe :] Un mot de passe qui permet à l'utilisateur de protéger son compte.
\end{description}
Après la soumission de ce formulaire, une recherche dans la base de données est faite afin de trouver s'il existe une personne correspondant au couple (login, mot de passe) donnée par l'utilisateur.
Dans le cas contraire, un message d'erreur lui est transmis. Nous restons tout de même vague sur la nature de l'erreur afin de protéger le compte d'une personne mal intentionnée.
Si la personne n'a pas de compte, elle est alors invitée à s'inscrire via le lien disponible sur cette même page. 

\section{Inscription}
L'inscription va permettre à tout utilisateur de créer un compte TwitIIE. Il aura alors ensuite accès a toutes les fonctionnalités de l'application.
Pour cela, quelques informations lui sont demandé via un formulaire de type POST. Les informations suivantes lui seront demandées : 
\begin{description}
\item[Login :] Un login, qui est unique et qui permettra à l'utilisateur de se connecter. Si le login choisi n'est pas disponible, un message d'erreur lui est envoyé afin q'il puisse le modifier.
\item[Mot de passe :] Afin de garantir de la sécurité de son compte, un mot de passe est requis.
\item[Confirmation de mot de passe :] Une confirmation du mot de passe précedement entré est demandé afin d'éviter toute faute de frappe.
\item[Nom :] Afin de renseigner son compte le nom de famille est demandé.
\item[Prenom :] De même, le prénom est demandé.
\item[Date de naissance :] Permet également de renseigner le profil de l'utilisateur. On pourrait également faire un test de majorité, mais l'application TwitIIE ne présente pour le moment aucun danger pour les personne mineures, l'accès leur est donc autorisé.
\end{description}


\section{La modification du profil}
L'utilisateur a la possibilité de modifier son profil quand il le souhaite, pour cela il doit se rendre sur la page Modifier\_profil.php accessible depuis la barre de navigation.
L'utilisateur est ensuite invité à remplir un formulaire de modification contenant les champs suivants :
\begin{description}
\item[Ancien mot de passe]
\item[Nouveau mot de passe]
\item[Confirmation du nouveau mot de passe]
\item[Nom]
\item[Prenom]
\item[Date de naissance]
\end{description}
Les champs Ancien mot de passe, Nom, Prenom et Date de naissance doivent etre remplis. Pour faciliter l'utilisation nous remplissons avec les valeurs actuelles ces champ (sauf Ancien mot de passe bien-sur).
Il n'a alors besoin de les modifier que s'il en a envie. Les champs sont alors mis a jours dans la base de données grace à un userManager. Nous mettons également à jour les variables de Session.



\section{Les Abonnements}	

Pour gérer les abonnements, on a crée la table \texttt{amis} composé de deux id, clés étrangères de la table \texttt{user} :
\begin{description}
\item[personne1 :] l’id de l’utilisateur qui s’abonne
\item[personne2 :] l’id de l’utilisateur à qui personne1 va s’abonner
\end{description}
C’est à l’aide de cette table que l’on récupère la liste des abonnements de l’utilisateur connecté, utile pour gérer ses messages ainsi que son fil d’actualité.
De plus, il est possible d’effectuer une recherche (via le formulaire \texttt{recherche @} pour accéder au profil d’un utilisateur inscrit. Un formulaire est alors proposé :
\begin{description}
\item  -si l’utilisateur n’est pas abonné à celui recherché, on lui propose d’ajouter celui ci dans sa liste d’abonnements
\item -sinon, on lui propose de supprimer son abonnement.
\end{description}

\section{Les Tweets}	

\subsection{Tweet}
Chaque utilisateur a possibilité d'écrire un tweet qui sera par la suite visible en allant sur son profil ou dans le fil d'actualité d'une personne abonnée. La table \texttt{tweet} est composée de :
\begin{description}
\item[id :] l'id du tweet
\item[auteur :] l’id de l’auteur du tweet
\item[date\_envoie :] la date du tweet
\item[contenu :] le contenu du tweet
\end{description}
Un formulaire sur la page \texttt{accueil} permet à l’utilisateur d’écrire un tweet.
De plus, on récupère les tweets des abonnements sur le fil d’actualité à l’aide d'une  jointure $n…n$ sur la table \texttt{amis} et \texttt{tweet}.

Lorsque l'utilisateur est un administrateur, il peut supprimer des tweets. Cela permet d'avoir une modération des publications.

\subsection{Likes}
Il est aussi possible d'aimer un tweet. On crée une table \texttt{like} spécifique à cette fonctionnalité, prenant en compte les informations suivantes :
\begin{description}
\item[tweet \_id :] l’id du tweet aimé
\item[user\_id:] l'id de l'utilisateur qui aime le tweet
\end{description}
On peut ainsi proposer à l'utilisateur d'aimer ou de ne plus aimer un tweet et récupérer le nombre de j'aime associé à un tweet.

\subsection{Liens cliquables}		
Pour chaque tweet, les noms d'utilisateurs @ et les hashtags \# sont convertis en liens cliquables. En effet, on peut ainsi directement accéder au profil d'un utilisateur ou encore à l'ensemble des tweets qui contiennent un hashtags. 

Pour réaliser cela, lors de l'affichage d'un tweet, le contenu du tweet est analysé mot par mot afin de remplacer tous les noms et hashtags par des liens utilisant la méthode GET. Il suffit en effet juste de trouver les mots commençant par \# ou @ puis de trouver l'id correspondant afin de l'inclure dans le lien.



\section{Les Commentaires}	
Chaque utilisateur à la possibilité de commenter un tweet. Il peut aussi commenter un commentaire. Nous aurons alors plusieurs niveau de commentaires. La principale difficulté rencontrée ici était de gérer ces différents niveaux de commentaires.

Voici comment nous avons décidé de créer notre table \texttt{commentaire}:
\begin{description}
\item[id :] l'id du commentaire;
\item[owner\_id :] l'id de l'auteur du commentaire
\item[target\_id :] l'id de l'auteur du tweet ou commentaire commenté
\item[date\_envoie :] la date du commentaire
\item[contenu :] le contenu du commentaire
\item[parent\_id :] l'id du tweet ou commentaire commenté
\item[parent\_type :] type du message commenté "tweet" ou "commentaire"
\end{description}
Nous avons aussi créer les classes \texttt{Commentaire} et \texttt{CommentaireManager} afin de manipuler plus facilement les commentaires.

Pour l'affichage des commentaires avec les différents niveaux, il a fallu une récursion. En effet, pour chaque commentaire, on va exécuter une nouvelle fois la requête pour récupérer tous les commentaires du commentaire courant.

Lorsque l'utilisateur est un administrateur, il peut supprimer des commentaires. 

\newpage
\section{Les Hashtags}		
L'utilisateur a la possibilité d'écrire des hashtags dans ses tweets, c'est-à-dire, des mots commençant par un \#. 

Voici comment est composée la table \texttt{hashtag}:
\begin{description}
\item[id :] l'id du hashtag;
\item[mot :] le contenu du hashtag
\end{description}
Pour relier chaque hashtag à des tweets, on a utilisée une table \texttt{hashtagEtTweet} : 
\begin{description}
\item[id\_hashtag :] l'id du hashtag;
\item[id\_tweet :] l'id du tweet;
\end{description}

L'ajout des hashtags dans la table est faite dès qu'un utilisateur envoie un tweet. Le contenu du tweet est analysé afin d'extraire tous les mots commençant par un \#. Ensuite, on regarde s'il existe déjà un hashtag similaire dans la table. S'il en existe un, le tweet est ajouté dans la table \texttt{hashtagEtTweet}, sinon on ajoute le hashtag dans la table \texttt{hashtag} dans un premier temps.
\vspace{1\baselineskip}

On a aussi la possibilité de rechercher tous les tweets contant un certain hashtag dans la barre de recherche \#. Il y a aussi des suggestions de hashtags lors de la recherche qu'on a réalisé grâce à AJAX.

%\newpage
\section{Les Messages Privés}	
Nous avons décider d'ajouter une fonctionnalité d'envoi de messages privés entre chaque utilisateur. 
Cette fonctionnalité est directement accessible depuis la barre de navigation. L'utilisateur peut alors choisir un destinataire parmi ses amis afin de lui envoyer un message privé.

Pour réaliser cela, nous avons dû créer une nouvelle table dans la base de données, voici ses attributs :

 \texttt{message(id, emetteur, recepteur, date\_envoie, contenu)}\\		
On a aussi décidé de créer une classe \texttt{Message} et une classe \texttt{MessageManager} afin de faciliter la manipulation des messages. 

\vspace{2\baselineskip}

Pour cette partie, on a principalement utilisé de l'AJAX pour envoyer des messages privés et pour charger les messages reçus automatiquement sans avoir besoin de réactualiser la page. Ces deux éléments sont les principales difficultés rencontrées pour réaliser cette fonctionnalité.



\section{Le style de l'application}
Pour la mise en page de notre application nous avons utilisé le CSS. La difficulté rencontrée ici a été de pouvoir se projeter et de s'accorder sur les couleurs et le style pour toutes les pages. Nous n'avons donc dans un premier temps pas pris compte de la mise en page de l'application pour se concentrer sur le fonctionnement de celle-ci.
Une fois le projet avancé, nous avons réalisé des maquettes de toutes les pages afin d'avoir une idée plus claire de l'agencement de celles-ci. Nous avons ensuite choisi les couleurs de l'application grace au site color.adobe.com qui propose un panel de couleurs à associer.


%\newpage		
\section{Répartition des tâches}

Kevin s'est principalement occupé de la partie des messages privés. Julien était en charge des tweets et des abonnements. Ensuite, Jiahui a travaillé sur les hashtags et commentaires avec Kevin. Edwin a réalisé l'inscription, la connexion et la modification de profil de l'application web. Jiahui s'est chargé de la partie administrateur.

Nous avons réalisé ensemble des maquettes pour toutes les pages présentes sur le site. C'est comme cela que le style général a été choisi. Chaque personne s'est ensuite occupé du style de la page qu'il avait développé.

\section{Conclusion}		

Ce projet nous permis de mettre à profit les connaissances acquises durant les cours de web. Cela nous a été utile d'avoir une première expérience en développement web. 




Malheureusement, nous n'avons pas réalisé la fonctionnalité de notifications et par manque de temps. 

\end{document}
 
