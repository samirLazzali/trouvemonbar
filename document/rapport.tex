\documentclass{article}
\usepackage[utf8]{inputenc}
\usepackage{program}
\usepackage{listings}

\author{Groupe CATisfaction}
\title{La création de CATisfaction : Le site de rencontre pour nos amis félins}

\begin{document}
\pagenumbering{gobble}
\maketitle
\newpage

\pagenumbering{gobble}
\tableofcontents
\newpage

\pagenumbering{arabic}

\section{Introduction}

\subsection{Le groupe}
Le groupe "CATisfaction" est composé des étudiants suivants : 
\begin{itemize}
\item Thomas "Psiich0" BERATTO 
\item Corentin "Tuareg" LAFONT
\item Quentin "Docteur" DELAVELLE
\item Adrien "Fontaine" VIZIER
\end{itemize}

\subsection{Le projet}
On connaît tous \textbf{Tinder}, \textbf{Meetic} ou les autres sites de rencontres. A l'heure où le numérique se développe et où les sorties se raréfient, ces sites sont de plus en plus plébiscités par une population de tous les âges. \textbf{CATisfaction} s'inspire de ce succès pour l'adapter à un tout autre domaine. En effet, \textbf{CATisfaction} reconditionne le concept de match et de recherche de partenaire pour l'appliquer au monde félin. Le site s'adresse à toutes ces personnes qui aiment trop leur compagnon pour envisager de le stériliser, et qui souhaitent, parce qu'elles n'en peuvent plus des miaulements de leur femelle en chaleur, ou bien parce qu'elles recherchent le partenaire idéal pour conserver la lignée de leur chat de race.
\newline
\newline
Premier mondial à se lancer dans le secteur, \textbf{CATisfaction} répond à une recherche qui ne faisait jusqu'alors que de bouches à oreilles ou bien entre éleveurs associés pour l'élargir au niveau mondial grâce à Internet.

\newpage
\section{Réalisation du site}

\subsection{Architecture générale}


\subsection{Système de matching}

\subsection{Design}

\section{Pour aller plus loin}

\end{document}